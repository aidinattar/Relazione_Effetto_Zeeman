
\documentclass{article}  
\usepackage{graphicx}
\usepackage[utf8]{inputenc}
\usepackage[T1]{fontenc}
\usepackage{float}
\usepackage[italian]{babel}
\usepackage{listings}
\usepackage[usenames]{color}
\usepackage{natbib}
\usepackage{siunitx}
\usepackage[strict]{changepage}
\usepackage{physics}
\usepackage{wrapfig}
\usepackage[a4paper, top=2cm, bottom=2cm, right=2cm, left=2cm]{geometry}
\usepackage{array}
\usepackage{color}
\usepackage{colortbl}
\usepackage{amsmath}
\usepackage{amssymb}
\usepackage{multirow}
\usepackage{enumitem}
\usepackage{hyperref}
\usepackage{times}
\usepackage{booktabs}
\usepackage{subfig}
\usepackage{multirow}



\title{Effetto Zeeman}
\author{Docente: dott. Garfagnini, dott. Lunardon \\
	Gruppo 14}
\date{Anno accademico 2019/20}

\begin{document}
	
	
	
	\maketitle
	
	\begin{itemize}
		\item[$\circ$] Aidin Attar - 1170698 - aidin.attar@studenti.unipd.it
		\item[$\circ$] Ema Baci - 1171107 – ema.baci@studenti.unipd.it
		\item[$\circ$] Alessandro Bianchetti – 1162147 – alessandro.bianchetti@studenti.unipd.it
	\end{itemize}
	
	\vspace{3 cm}
	\begin{large}\textsc{\textbf{Scopo dell'esperienza}: studio dell'effetto Zeeman per atomo di Neon, stima del fattore di Landè} 
	\end{large}
	\vspace{8.5cm}
	
	\begin{figure}[H]
		\centering
		\section{Conclusioni}\includegraphics[scale = 0.5 , angle=0]{unipd_logo.png}
	\end{figure}
	
	%\newpage \tableofcontents \newpage
	
	\twocolumn
	
	\subsection*{Descrizione dell'apparato} 
	
	L'apparato sperimentale si compone di una lampada al Neon, capace di emettere
	radiazione elettromagnetica associta principlamente a transizioni tra 3p e 2p.
	In particolare siamo interessati alla transizione da $\lambda = $585.3 nm.
	Tale lampada è inserita all'interno di una cavità magnetica, per cui ci si 
	aspetta la comparsa di righe di emissione: tuttavia, considerando la 
	polarizzazione delle righe, possiamo “sopprimere” la transizione centrale
	orientando il campo magnetico in modo longitudinale alla linea di 
	osservazione, in modo tale da vedere solo le due righe marginali. 
	
	Il raggio di luce emessa passa quindi per una lente condensante, necessaria 
	per concentrare quanto più possibile il fascio, che passa quindi per una 
	fenditura e successivamente attraverso un prisma, necessario per ruotare di 
	$90\deg$ la luce e dividerla nelle sue componenti principali. 
	
	A questo punto i raggi incidono sulla lamina di Lummer-Gehrcke, dispositivo
	interferenziale ad alta risoluzione, in modo quasi
	radente (faremo in seguito una stima dell'angolo di incidenza) e il fascio 
	emergente viene infine focalizzato sul dispositivo ottico di lettura (un CCD
	monodimensionale) attraverso la lente di camera.
	
	%La presa dati di questa esperienza è stata svolta interamente da remoto; in 
	%particolare la connessione al PC del laboratorio avviene tramite 
	%l'esportazione dello schermo via VNC Viewer.
	
	L'acquisizione dei dati è avvenutatramite l'interfaccia di controllo dell'
	apparato, che permette di controllare le diverse componenti appena citate.
	
	Per l'analisi dati sono stati utilizzati programmi scritti in c++, root ed Excel.
	
	%\begin{center}
	%\begin{figure}[ht]
	%\centering
	%\includegraphics[scale=0.28, angle=0]{alimentazione.pdf}
	%\caption{ schema per l'alimentazione dell'operazionale }
	%\label{fig:alimentazione}
	%\end{figure}
	%\end{center}
	
	\subsection*{Compendio di teoria}
	
	L'esperienza è incentrata sull'identificare e misurare lo splitting 
	Zeeman dei livelli di energia e confrontarli con le previsioni teoriche.
	L'effetto Zeeman è il fenomeno che si verifica nel momento in cui gli
	atomi di una certa sostanza vengono sottoposti a un campo magnetico 
	esterno, che suddivide i livelli energetici permessi per gli elettroni.
	In tal modo confrontando lo spettro della prima riga di transizione 
	($\lambda = $585.3 nm, associata ai termini spettroscopici
	$^1S^0 \xrightarrow[]{}  ^1P^1$ ) a campo magnetico spento e a campo 
	acceso vedremo 
	il biforcarsi delle campane in due ulteriori picchi. Misurando la
	distanza che separa i due picchi causati dalla presenza del campo,
	otterremo il doppio dello splitting $\Delta\lambda_{Zee} $.
	
	In particolare, si sceglie la transizione indicata sopra perché connette
	stati con spin nullo e con $\Delta L = 1$, per cui ci assicura l'effetto
	Zeeman cosiddetto \textit{normale}, cioè quello previsto anche dalla 
	teoria classica, dove il fattore giromagnetico presente nella formula 
	generale (effetto Zeeman \textit{anomalo}) sarà dunque semplicemente
	$g_l = 1$. Pertanto
	
	\[
	\Delta E_{Zee} = g_l m \mu_B B = \pm \mu_B B
	\]
	
	dato che nel nostro caso le due proiezioni del momento angolare totale
	dello stato di arrivo possibili sono $m = \pm 1$ (0 è escluso grazie all'
	orientazione del campo magnetico, come accennato sopra).
	
	Inoltre
	
	\[
	\Delta\lambda_{Zee} = \frac{\lambda^2}{hc}\Delta E_{Zee}
	\]
	
	rappresenta la formula con cui confronteremo il dato sperimentale
	ricavato.
	
	Inoltre da opportuni fit dei picchi ricaveremo una stima della larghezza
	delle campane, da cui ricaveremo il potere risolvente come rapporto
	
	\[
	R = \frac{\lambda}{\Delta\lambda}    
	\]
	
	che sarà ben coperto dalla risoluzione dell'apparato, cioè dalla
	risoluzione offerta dalla lamina di Lummer-Gehrcke, che segue l'ordine
	del rapporto L/$\lambda$, dove L è la lunghezza della lamina.
	Un altro parametro importante della lamina è il \textit{range di 
		lunghezza d'onda utile} $\Delta\lambda_{r.u.}$, che rappresenta la
	massima differenza tra due lunghezze d'onda tale che i due massimi 
	consecutivi restino distinguibili. In particolare
	
	\[
	\Delta\lambda_{r.u.} = \frac{\lambda^2}{2d}\frac{\sqrt{n^2-\sin(i)^2}}{n^2-\sin(i)^2-n\lambda\frac{dn}{d\lambda}}
	\]
	
	dove la derivata $dn/d\lambda$ è stata ottenura fittando alcuni punti 
	n($\lambda$) rappresentativi dell'andamento dell'indice di rifrazione
	della lamina in funzione della lunghezza d'onda. Nello specifico si è
	scelto il modello della legge di Cauchy fermandosi al primo ordine:

\[
n(\lambda) = A + \frac{B}{\lambda^2}    
\]

ottenendo il seguente andamento:
%\begin{center}
%\begin{figure}[ht]
%\centering
%\includegraphics[scale=0.28, angle=0]{nFit}
%\caption{ Indice di rifrazione }
%\label{fig:alimentazione}
%\end{figure}
%\end{center}

I parametri del fit sono :
$$ A= 1.4997 \pm 0.0009 \quad B= (42 \pm 3 )10^4 \quad m^2$$
 Per $\lambda= 585.3 nm$ si ottiene quindi :
$$n(\lambda)= 1.512 \pm 0.001  $$
$$  \frac{dn}{d\lambda} = (-4.19 \pm 0.03 )\cdot 10^{-5} $$

\section*{Aquisizione ed analisi dati}
\paragraph{Calibrazione}
Prima di procedere con la presa dati si è eseguita una calibrazione 
dell'apparato; in particolare occorre regolare la posizione 
della sorgente rispetto all'apertura della fenditura.
Il primo passaggio è stato quindi impostare l'apertura della fenditura 
a $24 \mu m$ ed aquisire un primo spettro a lamina disinserita, con il 
CCD orizzontale. Zoomando sui primi 3/4 picchi dello spettro, saremo
in grado di individuare il nostro picco di interesse.
Successivamente si regola ulteriormente la posizione della sorgente 
cercando quella che massimizzi l'intensità e la forma del picco di 
interesse.
Si è quindi fissata la posizione del CCD a 0.93 mm e impostato l'apertura 
della fenditura a $18 \mu m$.
Infine si sono regolate le posizioni della lente focale e della lente 
condensante, scegliendo come posizioni ottimali rispettivamente 15.36 mm 
e 14.62 mm.


\subsection*{Campo magnetico spento}

Dopo la calibrazione si può proseguire con l'acquisizione del primo 
spettro a campo magnetico spento: dopo aver selezionato il picco di 
interesse si inserisce la lamina di Lummer-Gerhcker, si seleziona 
con i cursori il range dello scanning e si ruota la posizione del CCD
in verticale.
Infine si imposta un tempo di integrazione di 600 ms e si avvia lo 
scanning, dopo aver salvato il file nell'opportuna cartella.


\paragraph{Analisi dello spettro}
Per procedere con l'analisi dello spettro, si procede caricando su root la macro ReadZeemanImage.C++ ed eseguendola. Si ottiene quindi un istogramma bidimensionale, al quale si sottrae il segnale di fondo riferendosi alla sua proiezione sull’asse x.
Dopo aver sottratto il fondo, migliorando la nitidezza dell'immagine, si registrano i limiti della zona del segnale. 
Si ottiene quindi anche la proiezione di tale regione sull’asse y, in cui si osservano una serie di picchi, di cui si migliora la pulizia dell'immagine mediante un rebin.

\paragraph{Analisi dei picchi}


Per fornire un'analisi estesa dello spettro si è scelto di raggruppare
i picchi in 6 terne di picchi consecutivi, e per ciascuna terna
l'obiettivo è studiare le spaziature tra i picchi e caratterizzarne in
qualche modo la larghezza. 
In particolare si è scelto di eseguire un fit gaussiano sui picchi
per poter individuare con precisione l'ordinata di massimo e per
poter dare una stima della larghezza del picco come FWHM.

Per prima cosa, studiamo l'andamento delle spaziature tra i picchi. 
Per ciascuna terna, si sono calcolate le due distanze tra i valori medi
delle tre gaussiane, considerandone poi la media aritmetica, disponendo
così di un valore univoco per caiscuna terna.

In particolare ci si aspetta un andamento quadratico in funzione della posizione del CCD.





\section*{Conclusioni}



\newpage
\appendix
\section{Appendici}
\label{appendice}
\subsection{Costruzione dell'errore sulle misure}
\label{Calcerr}

\subsection{Tabella delle compatibilità}
\medskip
\begin{table}[H]
    \centering
    \begin{tabular}{c}
        %\hline
        \begin{Large}
        $\lambda=\frac{|a-b|}{\sqrt{\sigma_a^2+\sigma_b^2}}$
        \end{Large}\\
        %\hline
    \end{tabular}
    \hspace{0.5cm}
    \begin{tabular}{cc}
        \toprule
        &       \textbf{Compatibilità   }       \\
        \midrule
        0$\leq \lambda$<1   &Ottima                 \\
        1$\leq \lambda$<2   &Buona                  \\
        2$\leq \lambda$<3   &Accettabile            \\
        3$\leq\lambda$<5   &Pessima                \\
        $ \lambda \geq $  5     &Non compatibile        \\
        \bottomrule
    \end{tabular}
    \caption{indicazioni lettura compatibilità}
    \label{tab:compatibilità}
\end{table}

\subsection{Dati sperimentali}


   
\end{document}
