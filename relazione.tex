\documentclass{article}  
\usepackage{graphicx}
\usepackage[utf8]{inputenc}
\usepackage[T1]{fontenc}
\usepackage{float}
\usepackage[italian]{babel}
\usepackage{listings}
\usepackage[usenames]{color}
\usepackage{natbib}
\usepackage{siunitx}
\usepackage[strict]{changepage}
\usepackage{physics}
\usepackage{wrapfig}
\usepackage[a4paper, top=2cm, bottom=2cm, right=2cm, left=2cm]{geometry}
\usepackage{array}
\usepackage{color}
\usepackage{colortbl}
\usepackage{amsmath}
\usepackage{amssymb}
\usepackage{multirow}
\usepackage{enumitem}
\usepackage{hyperref}
\usepackage{times}
\usepackage{booktabs}
\usepackage{subfig}
\usepackage{multirow}



\title{Effetto Zeeman}
\author{Docente: dott. Garfagnini, dott. Lunardon \\
Gruppo 14}
\date{Anno accademico 2019/20}

\begin{document}



\maketitle

    \begin{itemize}
        \item[$\circ$] Aidin Attar - 1170698 - aidin.attar@studenti.unipd.it
        \item[$\circ$] Ema Baci - 1171107 – ema.baci@studenti.unipd.it
        \item[$\circ$] Alessandro Bianchetti – 1162147 – alessandro.bianchetti@studenti.unipd.it
    \end{itemize}

\vspace{3 cm}
\begin{large}\textsc{\textbf{Scopo dell'esperienza}: studio dell'effetto Zeeman per atomo di Neon} 
\end{large}
\vspace{8.5cm}

%\begin{figure}[H]
%\centering
%\includegraphics[scale=0.5, angle=0]{unipd_logo.png}
%\end{figure}

%\newpage \tableofcontents \newpage

\twocolumn

\subsection*{Descrizione dell'apparato} 

L'apparato sperimentale si compone di una lampada al Neon, capace di emettere
radiazione elettromagnetica associta principlamente a transizioni tra 3p e 2p.
In particolare siamo interessati alla transizione da $\lambda = $585.3 nm, 
associata ai termini spettroscopici
$^1S^0 \xrightarrow[]{}  ^1P^1$ 
Tale lampada è inserita all'interno di una cavità magnetica, per cui ci si 
aspetta la comparsa di righe di emissione: tuttavia, considerando la 
polarizzazione delle righe, possiamo “sopprimere” la transizione centrale
orientando il campo magnetico in modo longitudinale alla linea di 
osservazione, in modo tale da vedere solo le due righe marginali. 

Il raggio di luce emessa passa quindi per una lente condensante, necessaria 
per concentrare quanto più possibile il fascio, che passa quindi per una 
fenditura e successivamente attraverso un prisma, necessario per ruotare di 
$90\deg$ la luce e dividerla nelle sue componenti principali. 

A questo punto i raggi incidono sulla lamina di Lummer-Gehrcke in modo quasi
radente (faremo in seguito una stima dell'angolo di incidenza) e il fascio 
emergente viene infine focalizzato sul dispositivo ottico di lettura (un CCD
monodimensionale) attraverso la lente di camera.

%La presa dati di questa esperienza è stata svolta interamente da remoto; in 
%particolare la connessione al PC del laboratorio avviene tramite 
%l'esportazione dello schermo via VNC Viewer.

L'acquisizione dei dati è avvenutatramite il l'interfaccia di controllo dell'
apparato, che permette di controllare le diverse componenti appena citate.

Per l'analisi dati sono stati utilizzati programmi scritti in c++, root ed Excel.

%\begin{center}
%\begin{figure}[ht]
%\centering
%\includegraphics[scale=0.28, angle=0]{alimentazione.pdf}
%\caption{ schema per l'alimentazione dell'operazionale }
%\label{fig:alimentazione}
%\end{figure}
%\end{center} 

\section{Conclusioni}



\newpage
\appendix
\section{Appendici}
\label{appendice}
\subsection{Costruzione dell'errore sulle misure}
\label{Calcerr}

\subsection{Tabella delle compatibilità}
\medskip
\begin{table}[H]
    \centering
    \begin{tabular}{c}
        %\hline
        \begin{Large}
        $\lambda=\frac{|a-b|}{\sqrt{\sigma_a^2+\sigma_b^2}}$
        \end{Large}\\
        %\hline
    \end{tabular}
    \hspace{0.5cm}
    \begin{tabular}{cc}
        \toprule
        &       \textbf{Compatibilità   }       \\
        \midrule
        0$\leq \lambda$<1   &Ottima                 \\
        1$\leq \lambda$<2   &Buona                  \\
        2$\leq \lambda$<3   &Accettabile            \\
        3$\leq\lambda$<5   &Pessima                \\
        $ \lambda \geq $  5     &Non compatibile        \\
        \bottomrule
    \end{tabular}
    \caption{indicazioni lettura compatibilità}
    \label{tab:compatibilità}
\end{table}

\subsection{Dati sperimentali}


   
\end{document}